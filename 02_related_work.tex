\section{Related Work}
\textbf{Rigging \& rig space}
Rigging can be tackled in two different directions. 
First, several studies have addressed the issue of automatic rigging from scratch for arbitrary characters \cite{baran2007automatic,borosan2012rigmesh,bang2015interactive}. 
Another line of research focuses on transferring the rigging of a source character to a similar target character\cite{poirier2009rig,seo2010rigging}. Our method is different from both of these approaches in that we focus on the transfer of motion into rig controllers for intuitive follow-up editing.
The term rig space was first introduced in Hahn et al.\shortcite{hahn2012rig,hahn2013efficient}, in which secondary deformations of a character mesh were simulated, and the results were mapped to the rig space parameters of the character to allow for controllable deformation. 
Our motivation is similar to theirs in that the aim is to achieve deformation that allows follow-up editing easy. 
Whereas they focused on deformation of mesh vertices of the same characters, we deal with hierarchical skeletons of the rig of an arbitrary character to transfer the motion.
%The most similar work to ours is \cite{holden2015learning}.
The work of Holden et al.\shortcite{holden2015learning} is the most similar to ours. 
They defined a function that describes the relationship between the rig parameters and the skeletal parameters and solved the inverse function using non-linear regression based on the keyframe data provided by the artist. 
Although this approach works well with proper examples, preparing the data per each rig type can be tedious, and the results can be biased toward the provided example data. 
Our spacetime optimization for motion mapping does not require any example data.

\textbf{Skeletal pose representation}
We compare the distance between the input skeletal pose and the rig skeletal pose during the optimization. 
Therefore, the skeletal pose representation that allows accurate measurement of the distance between them is required. 
To represent a human pose, Ho et al.\shortcite{ho2010spatial} and Holden et al.\shortcite{holden2015learning} utilized joint positions while Kulpa et al.\shortcite{kulpa2005morphology} relied on the normalized parameters. 
Unfortunately, these skeletal pose representation is not guaranteed to be a closed set in the rigid body skeleton configuration space for transition. 
Therefore, the energy function cannot be represented accurately in our optimization. 
R. Vemulapalli et al.\shortcite{vemulapalli2014human} defined a human pose using the relationship between each rigid body segment, and represented the human motion as a curve on the Lie group \SE{}. 
We also employ the representation based on the Lie group \SE{}. 
However, to handle the inverse rig mapping problem, we define the skeletal pose as a serial transformation of the hierarchical rigid segments along with the global positions of each rigid segment.

\textbf{Motion retargeting}
Our work touches on motion retargeting. 
Many studies tackled motion retargeting based on joint configurations of the human body. Gleicher\shortcite{gleicher1998retargetting} proposed an offline method that retargets the motion of the source joints to the target joints. 
Choi and Ko\shortcite{choi1999line} and Shin et al.\shortcite{shin2001computer} provided online retargeting solutions that utilized inverse kinematics. 
The results of these methods are parameters defined in the joint space of the target character, making subsequent modification of the retargeted motion difficult. 
Hecker et al.\shortcite{hecker2008real} utilized the meta data defined from the motion clips authorized by their semantic tools to retarget to a variety of morphology independent characters. 
Unfortunately, this method is applicable only to the characters created in their animation system. 
Seol et al.\shortcite{seol2013creature} proposed a learning model of motion data pairs between human and non-human characters to create the puppetry motion from a new input human motion. 
Although they allow the mapping of input motion to target control handles for keyframing, their method is limited to motion data pairs. 
In contrast, our goal is to find the optimal rig space parameters for a given input motion. 
Thus, we perform the optimization using an identical skeleton structure between the source motion and the target rigged character for visually correct results. 
To use existing motion data from a character that has different topology or proportion of the skeleton as input, we initially perform an existing motion retargeting method\cite{palamar2013mastering} to transfer the motion to the same skeleton structure as the target rig.

%\subsection{prev}
%\textbf{Rigging \& Rig space}
%Rigging is one of the important areas related to our study. Recently, several studies have addressed the issue of considered ways to develop automatic rigging for arbitrary characters [Baran and Popovi´c 2007][Pan et al. 2009] [Bharaj et al. 2012]. Another line of research focuses on transferring the rigging of a source character to a similar target character [Poirier and Paquette 2009][Seo et al. 2010]. Our method is different from these in that we focus on the transfer of resulting rig controllers for intuitive follow-up editing.
%
%The term “Rig-space” was first introduced in Hahn et al. [2012][2013], in which secondary deformations of a character mesh were simulated and the results were mapped to the rig-space parameters of the character to allow for controllable deformation. Our motivation is similar to theirs in that the aim is to achieve deformation that is easy for the artist to process. Whereas Hahn et al. focused on deformation of mesh vertices of the same characters, we use hierarchical skeletons of arbitrary character rig to transfer the motion.
%
%The most similar work to ours is [Inverse Rig Mapping, 2015]. They defined the function that describes the relationship between the rig parameter and the skeletal parameter and solved the inverse function using non-linear regression based on the animator’s keyframe data. However, such example based regression model requires example data (made by artist) per each rig type, and the result can be biased depending on the example data. Since we perform the motion mapping to rig using the spacetime optimization without any example data, our result is unbiased to specific artist or example data.
%
%\textbf{Skeletal pose representation}
%Since we compare the distance between the input pose and the rig generated pose during the optimization, 
%the Skeletal pose representation that can measure the accurate distance is required.
%To represent the human pose, \cite{ho2010spatial} and \cite{holden2015learning} used the joint position and \cite{kulpa2005morphology} used the normalized parameters. However, these pose representation is not guaranteed to be closed set (of rigid body skeleton configuration space) for transition. Therefore, it cannot be accurate energy function in our optimization problem. R Vemulapalli et al[2014] defined the human pose as the relationship between each rigid body part, and represented the human motion as curve on the Lie group $SE(3)\times ... \times SE(3)$. In this paper we also use relative rigid body representation based on the Lie Group $SE(3) \times ... \times SE(3)$. However, since we are handling the motion mapping problem, we additionally defined the important rigid segments that requires careful global transformation (ex. supporting foot). 
%Because our rigid body representation is classified body part based on rig operation, we can also effectively perform the optimization on non human character.
%
%\textbf{Motion Retargeting}
%From the point of view of our work is transferring the existing motion into the other character, it’s interest similar to other retargeting research. Many studies have considered motion retargeting techniques based on the joint configuration of the human body. Gleicher[1998],Witkin and Popovic[1995], Lee and Shin[1999] used offline methods to retarget source joint motion to targeted joints. Choi and Ko[1999] and Shin et al.[2001] provided online retargeting solutions that use inverse kinematics (IK). These approaches only retarget to topologically-identical target joints. Monzani et al.[2000] suggested the use of intermediate joints to retarget motions between a source and target that have topologically different joint structures. The results of these methods are parameters defined in the joint space of the target character, so subsequent modification of the retargeted motion by an animator is a difficult task.
%In this paper, our goal is find the optimal rig space parameter for given input motion. Thus, we performed the optimization using identical skeleton structure (on motion and rig character) for visually correct result. (to verify visual correctness?) To use the existing motion data with different topology or proportion skeleton structure as input, we initially performs the motion retargeting using \cite{palamar2013mastering} to transfer the existing motion to the same skeleton structure as the target rig. If we use other motion retargeting method, we can start our method from different starting point.